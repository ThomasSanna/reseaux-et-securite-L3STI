\documentclass{article}
\usepackage[utf8]{inputenc}
\usepackage{amsmath}
\usepackage{amsfonts}
\usepackage{amssymb}
\usepackage{graphicx}
\usepackage{indentfirst}
\usepackage{listings}
\usepackage{xcolor}
\usepackage{fancyhdr} % permet de creer des entetes et pieds de page


% Configuration de l'en-tête et du pied de page
\pagestyle{fancy}
\fancyhf{}
\lhead{Sanna Thomas}
\rhead{L3 STI}
\chead{Rapport Exercice 5: Réseaux et Sécurité}
\rfoot{Page n°\thepage}

% Configuration du package listings
\lstset{
  language=Python,
  basicstyle=\ttfamily\small,
  keywordstyle=\color{blue!80!black},
  commentstyle=\color{green!60!black},
  morecomment=[s][\color{gray!80}]{"""}{"""},
  numbers=left,
  numberstyle=\tiny\color{gray},
  stepnumber=1,
  numbersep=10pt,
  backgroundcolor=\color{gray!10},
  showspaces=false,
  showstringspaces=false,
  showtabs=false,
  frame=single,
  breaklines=true,
  breakatwhitespace=true,
  tabsize=4,
  captionpos=b,
  postbreak=\mbox{\textcolor{red}{$\hookrightarrow$}\space},
  inputencoding=utf8,
  extendedchars=true,
  literate={á}{{\'a}}1 {ã}{{\~a}}1 {é}{{\'e}}1 {è}{{\`e}}1 {ê}{{\^e}}1 {ë}{{\"e}}1 {û}{{\^u}}1 {ù}{{\`u}}1 {â}{{\^a}}1 {à}{{\`a}}1 {ç}{{\c{c}}}1 {Ç}{{\c{C}}}1 {É}{{\'E}}1 {Ê}{{\^E}}1 {È}{{\`E}}1 {Î}{{\^I}}1 {Ô}{{\^O}}1 {œ}{{\oe}}1 {Œ}{{\OE}}1 {æ}{{\ae}}1 {Æ}{{\AE}}1 {€}{{\euro}}1 {£}{{\pounds}}1 {«}{{\guillemotleft}}1 {»}{{\guillemotright}}1 {°}{{\textdegree}}1 {~}{{\textasciitilde}}1
}



\begin{document}

\begin{titlepage}
  \title{TD\\Socket Avancé\\Réalisation d'une Application de Chat basique}
  \author{SANNA Thomas}
  \date{\today}
  \maketitle
\end{titlepage}

\section{Introduction}

Ce rapport détaille la réalisation d'une application de chat basique permettant à plusieurs utilisateurs de se connecter simultanément. Le serveur utilise un modèle non-bloquant pour gérer les connexions et les messages. Le projet est divisé en trois parties principales:
\begin{enumerate}
    \item Encodage et décodage des messages.
    \item Implémentation du serveur.
    \item Implémentation du client.
\end{enumerate}

\section{Exercice 1 — Réalisation du Chat}

\subsection{Objectifs}

\begin{enumerate}
    \item \textbf{Structuration des messages}: Chaque message transmis contient un en-tête (un entier de 32 bits en big-endian) indiquant sa taille. Le récepteur doit d'abord récupérer cette valeur avant de lire le message complet.
    \item \textbf{Fonctions d'encodage et de décodage}:
    \begin{itemize}
        \item \texttt{encode\_message(mess: str)}: Encode le message en bytes, ajoute un timestamp et un en-tête contenant la taille du message.
        \item \texttt{decode\_message(encoded\_mess: bytes)}: Décode les bytes en extrayant l'en-tête et le message.
    \end{itemize}
\end{enumerate}

\break\subsection{Implémentation}

\subsubsection{Fichier \texttt{utils.py}}

Ce fichier contient les fonctions d'encodage et de décodage des messages.

\paragraph{Fonction \texttt{encode\_message}}:

\begin{lstlisting}[language=Python, caption=encode\_message dans utils.py]
import time
import struct

def encode_message(message: str) -> bytes:
    timestamp = time.strftime('%Y-%m-%d %H:%M', time.gmtime())
    full_message = f"{timestamp} {message}"
    message_bytes = full_message.encode('utf-8')
    header = struct.pack('>I', len(message_bytes))  # >I signifie que le format est big-endian unsigned int
    return header + message_bytes
\end{lstlisting}

\subparagraph{Explications}

\begin{itemize}
    \item \texttt{encode\_message}: Ajoute un timestamp au message, encode le message en bytes et ajoute un en-tête contenant la taille du message.
\end{itemize}

\paragraph{Fonction \texttt{decode\_message}}:

\begin{lstlisting}[language=Python, caption=decode\_message dans utils.py]
def decode_message(data: bytes) -> str:
    if len(data) < 4:
        raise ValueError("Data is too short to contain a valid message header")
    
    message_length = struct.unpack('>I', data[:4])[0]
    if len(data) < 4 + message_length:
        raise ValueError("Data is too short to contain the full message")
    
    full_message = data[4:4 + message_length].decode('utf-8')
    yearMonthDay, hourMinSec, message = full_message.split(' ', 2)
    return message
\end{lstlisting}

\subparagraph{Explications}

\begin{itemize}
    \item \texttt{decode\_message}: Décode les bytes en extrayant l'en-tête et le message. Vérifie que les données sont suffisamment longues pour contenir un message valide.
\end{itemize}

\section{Exercice 2 — Implémentation du Serveur}

\subsection{Objectifs}

\begin{enumerate}
    \item Sauvegarder les sockets acceptés dans une liste.
    \item Afficher le message reçu dans la console avec l'adresse du socket source.
    \item Transmettre le message à tous les autres clients connectés, en incluant l'heure et l'adresse de l'émetteur dans l'en-tête du message.
\end{enumerate}

\subsection{Implémentation}

\subsubsection{Fichier \texttt{serv.py}}

Ce fichier contient l'implémentation du serveur.

\paragraph{Fonction \texttt{accept}}:

\begin{lstlisting}[language=Python, caption=accept dans serv.py]
import selectors
import socket
import utils

# Create a default selector
sel = selectors.DefaultSelector()

# List to store accepted sockets
clients = []

def accept(sock, mask):
    conn, addr = sock.accept()  # Should be ready
    print("Accepted connection from", addr)
    conn.setblocking(False)
    sel.register(conn, selectors.EVENT_READ, read)
    clients.append((conn, addr))  # Add the new connection to the clients list
\end{lstlisting}

\subparagraph{Explications}

\begin{itemize}
    \item \texttt{accept}: Accepte une nouvelle connexion, l'ajoute à la liste \texttt{clients} et l'enregistre pour surveiller les événements de lecture.
\end{itemize}

\paragraph{Fonction \texttt{read}}:

\begin{lstlisting}[language=Python, caption=read dans serv.py]
def read(conn, mask):
    data = conn.recv(1000)  # Should be ready
    addr = conn.getpeername()
    if data:
        message = utils.decode_message(data)
        print(f"Received message from {addr}: {message}")

        # Create the message to be sent to other clients
        encoded_message = utils.encode_message(message)

        # Send the message to all other clients
        for client, client_addr in clients:
            if client != conn:
                client.send(encoded_message)
    else:
        print("Closing connection to", conn)
        sel.unregister(conn)
        conn.close()
        clients.remove((conn, addr))  # Remove the connection from the clients list
\end{lstlisting}

\subparagraph{Explications}

\begin{itemize}
    \item \texttt{read}: Lit les données du socket, affiche le message reçu avec l'adresse du socket source, et transmet le message à tous les autres clients connectés.
\end{itemize}

{Si \texttt{data} n'existe pas (c'est-à-dire que \texttt{recv} retourne une chaîne vide), cela signifie que le client a fermé la connexion. Dans ce cas, le serveur ferme la connexion avec le client, désenregistre le socket du sélecteur et le retire de la liste des clients.}

\break\paragraph{Configuration du Serveur et Boucle d'Événements}:

\begin{lstlisting}[language=Python, caption=Configuration du serveur et boucle d'événements dans serv.py]
# Create and bind the socket
sock = socket.socket()
sock.bind(("localhost", 12345))  # Change port to 12345 to match the client
sock.listen(100)
sock.setblocking(False)
sel.register(sock, selectors.EVENT_READ, accept)

print("Server started on port 12345")

# Event loop
while True:
    events = sel.select()
    for key, mask in events:
        callback = key.data
        callback(key.fileobj, mask)
\end{lstlisting}

\subparagraph{Explications}

\begin{itemize}
    \item \textbf{Configuration du Serveur}: Crée et lie le socket, le configure pour le mode non-bloquant, et l'enregistre pour surveiller les événements de lecture.
    \item \textbf{Boucle d'Événements}: Sélectionne les événements prêts et appelle la fonction de rappel appropriée pour chaque événement.
\end{itemize}

\break\section{Exercice 3 — Implémentation du Client}

\subsection{Objectifs}

\begin{enumerate}
    \item Créer une application client permettant de se connecter au serveur.
    \item Utiliser des threads pour gérer l'envoi et la réception des messages en parallèle. Cette fonctionnalité est utile car l'envoi et la réception de messages sont normalement des opérations bloquantes; Il est donc nécessaire d'utiliser des threads pour les exécuter simultanément.
\end{enumerate}

\subsection{Implémentation}

\subsubsection{Fichier \texttt{client.py}}

Ce fichier contient l'implémentation du client.

\paragraph{Fonction \texttt{send\_loop}}:

\begin{lstlisting}[language=Python, caption=send\_loop dans client.py]
import threading
import socket
import utils

# Define and connect the socket
sock = socket.socket(socket.AF_INET, socket.SOCK_STREAM)
sock.connect(('localhost', 12345))

def send_loop():
    while True:
        message = input()
        sock.send(utils.encode_message(message))
\end{lstlisting}

\subparagraph{Explications}

\begin{itemize}
    \item \texttt{send\_loop}: Lit les messages de l'utilisateur et les envoie au serveur.
\end{itemize}

\paragraph{Fonction \texttt{receive\_loop}}:

\begin{lstlisting}[language=Python, caption=receive\_loop dans client.py]
def receive_loop():
    while True:
        message = sock.recv(1000)
        print("Donnée envoyée par le serveur: ", utils.decode_message(message))
\end{lstlisting}

\subparagraph{Explications}

\begin{itemize}
    \item \texttt{receive\_loop}: Reçoit les messages du serveur et les affiche.
\end{itemize}

\paragraph{Utilisation des Threads}:

\begin{lstlisting}[language=Python, caption=Utilisation des threads dans client.py]
thread_send = threading.Thread(target=send_loop)
thread_send.start()

thread_receive = threading.Thread(target=receive_loop)
thread_receive.start()

thread_send.join()
thread_receive.join()
\end{lstlisting}

\subparagraph{Explications}

\begin{itemize}
    \item \textbf{Threads}: Utilise des threads pour exécuter \texttt{send\_loop} et \texttt{receive\_loop} simultanément.
\end{itemize}

\end{document}