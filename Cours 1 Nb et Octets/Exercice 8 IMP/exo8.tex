\documentclass{article}
\usepackage[utf8]{inputenc}
\usepackage{amsmath}
\usepackage{amsfonts}
\usepackage{amssymb}
\usepackage{graphicx}
\usepackage{indentfirst}
\usepackage{listings}
\usepackage{xcolor}
\usepackage{fancyhdr} % permet de creer des entetes et pieds de page


% Configuration de l'en-tête et du pied de page
\pagestyle{fancy}
\fancyhf{}
\lhead{Sanna Thomas}
\rhead{L3 STI}
\chead{Exercice 8}
\rfoot{Page n°\thepage}

\title{Rapport Exercice 8}
\author{Sanna Thomas, L3STI}
\date{\today}

\setlength{\parindent}{1cm}

\begin{document}

\maketitle

\section*{Contexte}

Pour transférer les données aux clients, le protocole de sérialisation binaire correspond
à la transmission de ces données, toutes encodées en big-endian, dans l’ordre:
\begin{itemize}
  \item Le numéro de signature de l’état (entier 32 bits non-signé);
  \item Le nombre d’utilisateurs (1 octet non-signé);
  \item Pour chaque utilisateur:
  \begin{itemize}
    \item son identifiant (1 octet non-signé);
    \item sa coordonnée sur x (short int signé);
    \item sa coordonnée sur y (short int signé).
  \end{itemize}
\end{itemize}
Interpréter le paquet reçu par le client ci-dessous:

\[00000000 \ 00000000 \ 00000000 \ 00011010\]
\[00000011 \ 10011101 \ 00000100 \ 11000000\]
\[10000000 \ 00001111 \ 01011001 \ 10000000\]
\[00011110 \ 10000000 \ 00000010 \ 11110110\]
\[00001000 \ 00000000 \ 00000000 \ 00001000\]

\section{Numéro de signature de l'état}

Le numéro de signature de l'état est codé sur 32 bits non-signés.
\[00000000 \ 00000000 \ 00000000 \ 00011010\]
Le numéro de signature de l'état est 26 en décimal.

\section{Nombre d'utilisateurs}

Le nombre d'utilisateurs est codé sur 1 octet non-signé.
\[00000011\] 
Cela donne 3 en décimal. Il y a donc 3 utilisateurs.

\section{Utilisateur 1}

\subsection{Identifiant}

L'identifiant de l'utilisateur est codé sur 1 octet non-signé.
\[10011101\]
Cela donne 157 en décimal. L'identifiant de l'utilisateur 1 est 157.

\subsection{Coordonnée sur x}

La coordonnée sur x de l'utilisateur est codée sur un short int signé soit 2 octets.
\[00000100 \ 11000000\]
Cela donne 1216 en décimal. La coordonnée sur x de l'utilisateur 1 est 1124.

\subsection{Coordonnée sur y}

La coordonnée sur y de l'utilisateur est codée sur un short int signé soit 2 octets.
\[10000000 \ 00001111\]
Cela donne -15 en décimal. La coordonnée sur y de l'utilisateur 1 est -15.

\section{Utilisateur 2}

\subsection{Identifiant}

L'identifiant de l'utilisateur est codé sur 1 octet non-signé.
\[01011001\]
Cela donne 89 en décimal. L'identifiant de l'utilisateur 2 est 89.

\subsection{Coordonnée sur x}

La coordonnée sur x de l'utilisateur est codée sur un short int signé soit 2 octets.
\[10000000 \ 00011110\]
Cela donne -30 en décimal. La coordonnée sur x de l'utilisateur 2 est -30.

\subsection{Coordonnée sur y}

La coordonnée sur y de l'utilisateur est codée sur un short int signé soit 2 octets.
\[10000000 \ 00000010\]
Cela donne -2 en décimal. La coordonnée sur y de l'utilisateur 2 est -2.

\section{Utilisateur 3}

\subsection{Identifiant}

L'identifiant de l'utilisateur est codé sur 1 octet non-signé.
\[11110110\]
Cela donne 246 en décimal. L'identifiant de l'utilisateur 3 est 246.

\subsection{Coordonnée sur x}

La coordonnée sur x de l'utilisateur est codée sur un short int signé soit 2 octets.
\[00001000 \ 00000000\]
Cela donne 2048 en décimal. La coordonnée sur x de l'utilisateur 3 est 2048.

\subsection{Coordonnée sur y}

La coordonnée sur y de l'utilisateur est codée sur un short int signé soit 2 octets.
\[00000000 \ 00001000\]
Cela donne 8 en décimal. La coordonnée sur y de l'utilisateur 3 est 8.




\end{document}