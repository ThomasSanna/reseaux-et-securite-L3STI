\documentclass{article}
\usepackage[utf8]{inputenc}
\usepackage{amsmath}
\usepackage{amsfonts}
\usepackage{amssymb}
\usepackage{graphicx}
\usepackage{indentfirst}
\usepackage{listings}
\usepackage{xcolor}


\lstset{
  language=Python,
  basicstyle=\ttfamily\small,
  keywordstyle=\color{blue!80!black},
  commentstyle=\color{green!60!black},
  morecomment=[s][\color{gray!80}]{"""}{"""},
  numbers=left,
  numberstyle=\tiny\color{gray},
  stepnumber=1,
  numbersep=10pt,
  backgroundcolor=\color{gray!10},
  showspaces=false,
  showstringspaces=false,
  showtabs=false,
  frame=single,
  breaklines=true,
  breakatwhitespace=true,
  tabsize=4,
  captionpos=b,
  postbreak=\mbox{\textcolor{red}{$\hookrightarrow$}\space},
  inputencoding=utf8,
  extendedchars=true,
  literate={á}{{\'a}}1 {ã}{{\~a}}1 {é}{{\'e}}1 {è}{{\`e}}1 {ê}{{\^e}}1 {ë}{{\"e}}1 {û}{{\^u}}1 {ù}{{\`u}}1 {â}{{\^a}}1 {à}{{\`a}}1 {ç}{{\c{c}}}1 {Ç}{{\c{C}}}1 {É}{{\'E}}1 {Ê}{{\^E}}1 {È}{{\`E}}1 {Î}{{\^I}}1 {Ô}{{\^O}}1 {œ}{{\oe}}1 {Œ}{{\OE}}1 {æ}{{\ae}}1 {Æ}{{\AE}}1 {€}{{\euro}}1 {£}{{\pounds}}1 {«}{{\guillemotleft}}1 {»}{{\guillemotright}}1 {°}{{\textdegree}}1 {~}{{\textasciitilde}}1
}

\title{Exercice 5 — Mini-projet de conception protocole}
\author{Sanna Thomas, L3STI}
\date{\today}

\begin{document}

\maketitle

\section{Introduction}

L'objectif de cet exercice est de concevoir un protocole de sérialisation utilisant le moins de mémoire possible pour enregistrer l'entrée d'une manette pour une frame. Les informations à enregistrer sont:
\begin{itemize}
    \item La position de deux joysticks analogiques (deux coordonnées à valeurs dans $[-1, 1]$ sur les axes x et y).
    \item L'appui de 9 boutons d'action différents.
\end{itemize}

La fonction de sérialisation retournera une instance de \texttt{Bytes}, résultat de la fonction \texttt{struct.pack}, et la fonction de désérialisation prendra en entrée une instance de \texttt{Bytes}. Le choix de la structure de données pour représenter la manette est libre.

\section{Protocole de Sérialisation}

On peut représenter les données de cette façon:
\begin{itemize}
    \item Chaque coordonnée des joysticks est représentée par un float (4 octets), sachant que chaque joystick a deux coordonnées entre -1 et 1.
    \item Les états des 9 boutons sont représentés par un entier non signé de 2 octets (16 bits), où chaque bit représente l'état d'un bouton (1 pour appuyé, 0 pour non appuyé).
\end{itemize}

Ainsi, la structure totale occupe $4 \times 4 + 2 = 18$ octets.

\section{Implémentation en Python}

\subsection{Fonction de Sérialisation}

La fonction de sérialisation convertit les états des boutons en un entier et empaquette les données en octets.

\begin{lstlisting}[caption=Fonction de Serialisation]
import struct

def serialiser(joystickGauche, joystickDroit, boutons):
    gaucheX, gaucheY = joystickGauche
    droitX, droitY = joystickDroit

    # convertir les etats des boutons en un seul entier
    boutonsInt = 0
    for i, appuye in enumerate(boutons):
        if appuye:
            boutonsInt += 2**i

    # empaqueter les donnees en octets
    donnees = struct.pack("!4fH", gaucheX, gaucheY, droitX, droitY, boutonsInt)
    return donnees
\end{lstlisting}

\subsection{Fonction de Désérialisation}

La fonction de désérialisation dépaquette les données des octets et convertit l'entier en une liste de booléens représentant l'état des boutons.

\begin{lstlisting}[caption=Fonction de Deserialisation]
def deserialiser(donnees):
    gaucheX, gaucheY, droitX, droitY, boutonsInt = struct.unpack("!4fH", donnees)

    # convertir l'entier en une liste de booleens
    boutons = []
    for i in range(9):
        masqueBit = boutonsInt & 2**i
        boutons.append(bool(masqueBit != 0))

    joystickGauche = (gaucheX, gaucheY)
    joystickDroit = (droitX, droitY)
    return joystickGauche, joystickDroit, boutons
\end{lstlisting}

\break\subsection{Exemple d'Utilisation}

\begin{lstlisting}[caption={Exemple d'Utilisation}]
# exemple
joystickGauche = (0.2, -0.5)
joystickDroit = (-1.0, 1.0)
boutons = [True, False, True, False, ...]

donneesSer = serialiser(joystickGauche, joystickDroit, boutons)
print(f"Donnees serialisees: {donneesSer}")

joyGDes, joyDDes, boutDes = deserialiser(donneesSer)
print(f"Joystick gauche deserialise: {joyGDes}")
print(f"Joystick droit deserialisee: {joyDDes}")
print(f"Boutons deserialises: {boutDes}")
\end{lstlisting}

\end{document}