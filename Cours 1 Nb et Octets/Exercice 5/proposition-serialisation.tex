\documentclass{article}
\usepackage[utf8]{inputenc}
\usepackage{amsmath}

\title{Protocole de Sérialisation}
\author{Sanna Thomas, L3STI}
\date{\today}

\begin{document}

\maketitle

\section*{Protocole de Sérialisation}

Pour enregistrer l'entrée d'une manette pour une frame en utilisant le moins de mémoire possible, nous proposons le protocole de sérialisation suivant:

\begin{itemize}
  \item \textbf{Position des joysticks analogiques}:
  \begin{itemize}
    \item Chaque joystick a deux coordonnées $(x, y)$ avec des valeurs dans l'intervalle $[-1, 1]$.
    \item Chaque coordonnée est représentée par un nombre flottant de 32 bits (4 octets) après quantification.
    \item \textbf{Exemple}: Si le joystick gauche est à la position $(0.2, -0.5)$ et le joystick droit à la position $(-1, 1)$, alors chaque coordonnée sera convertie en un nombre flottant de 32 bits.
  \end{itemize}
  \item \textbf{Appui des boutons d'action}:
  \begin{itemize}
    \item Il y a 9 boutons d'action différents.
    \item Chaque bouton est représenté par 1 bit (1 pour appuyé, 0 pour relâché).
    \item \textbf{Exemple}: Si les boutons 1, 3 et 5 sont appuyés, et les autres sont relâchés, cela sera représenté par la séquence binaire '101010000'.
  \end{itemize}
\end{itemize}

\section*{Organisation des Données}

\begin{itemize}
  \item 2 joysticks $\times$ 2 coordonnées $\times$ 4 octets = 16 octets pour les joysticks.
  \item 9 bits pour les boutons (arrondis à 2 octets pour simplifier l'alignement).
\end{itemize}

\section*{Exemple Complet}

Supposons que nous avons les entrées suivantes pour une frame:
\begin{itemize}
  \item Joystick gauche: $(0.2, -0.5)$
  \item Joystick droit: $(-1, 1)$
  \item Boutons appuyés: 1, 3, 5
\end{itemize}

\textbf{Étape 1: Sérialisation des joysticks}
\begin{itemize}
  \item Joystick gauche $(0.2, -0.5)$: chaque coordonnée est convertie en un nombre flottant de 32 bits.
    \begin{itemize}
      \item $0.2$ en binaire (4 octets): '00111110010011001100110011001101'
      \item $-0.5$ en binaire (4 octets): '10111111000000000000000000000000'
    \end{itemize}
  \item Joystick droit $(-1, 1)$: chaque coordonnée est convertie en un nombre flottant de 32 bits.
    \begin{itemize}
      \item $-1$ en binaire (4 octets): '10111111100000000000000000000000'
      \item $1$ en binaire (4 octets): '00111111100000000000000000000000'
    \end{itemize}
\end{itemize}

\textbf{Étape 2: Sérialisation des boutons}
\begin{itemize}
  \item Boutons appuyés: 1, 3, 5
  \item Séquence binaire: '101010000'
\end{itemize}

\section*{Total}

Au total, nous avons besoin de 16 octets pour les joysticks et 2 octets pour les boutons, soit un total de 18 octets pour représenter toutes les informations.

\textbf{Résultat binaire final}:
\begin{itemize}
  \item Joystick gauche $(0.2, -0.5)$:
  \begin{itemize}
    \item '00111110010011001100110011001101'
    \item '10111111000000000000000000000000'
  \end{itemize}
  \item Joystick droit $(-1, 1)$:
  \begin{itemize}
    \item '10111111100000000000000000000000'
    \item '00111111100000000000000000000000'
  \end{itemize}
  \item Boutons appuyés: '101010000' (arrondi à 2 octets: '10101000 00')
\end{itemize}

Le résultat binaire final est donc:
\[
00111110010011001100110011001101
\]
\[10111111000000000000000000000000\]
\[10111111100000000000000000000000\]
\[00111111100000000000000000000000\]
\[10101000 \ 00\]

\end{document}