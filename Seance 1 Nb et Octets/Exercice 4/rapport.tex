\documentclass{article}
\usepackage[utf8]{inputenc}
\usepackage{amsmath}
\usepackage{amsfonts}
\usepackage{amssymb}
\usepackage{graphicx}
\usepackage{indentfirst}
\usepackage{listings}
\usepackage{xcolor}
\usepackage{fancyhdr} % permet de creer des entetes et pieds de page


% Configuration de l'en-tête et du pied de page
\pagestyle{fancy}
\fancyhf{}
\lhead{Sanna Thomas}
\rhead{L3 STI}
\chead{Rapport Exercice 3: Réseaux et Sécurité}
\rfoot{Page n°\thepage}


% Configuration du package listings
\lstset{
  language=Python,
  basicstyle=\ttfamily\small,
  keywordstyle=\color{blue!80!black},
  commentstyle=\color{green!60!black},
  morecomment=[s][\color{gray!80}]{"""}{"""},
  numbers=left,
  numberstyle=\tiny\color{gray},
  stepnumber=1,
  numbersep=10pt,
  backgroundcolor=\color{gray!10},
  showspaces=false,
  showstringspaces=false,
  showtabs=false,
  frame=single,
  breaklines=true,
  breakatwhitespace=true,
  tabsize=4,
  captionpos=b,
  postbreak=\mbox{\textcolor{red}{$\hookrightarrow$}\space},
  inputencoding=utf8,
  extendedchars=true,
  literate={á}{{\'a}}1 {ã}{{\~a}}1 {é}{{\'e}}1 {è}{{\`e}}1 {ê}{{\^e}}1 {ë}{{\"e}}1 {û}{{\^u}}1 {ù}{{\`u}}1 {â}{{\^a}}1 {à}{{\`a}}1 {ç}{{\c{c}}}1 {Ç}{{\c{C}}}1 {É}{{\'E}}1 {Ê}{{\^E}}1 {È}{{\`E}}1 {Î}{{\^I}}1 {Ô}{{\^O}}1 {œ}{{\oe}}1 {Œ}{{\OE}}1 {æ}{{\ae}}1 {Æ}{{\AE}}1 {€}{{\euro}}1 {£}{{\pounds}}1 {«}{{\guillemotleft}}1 {»}{{\guillemotright}}1 {°}{{\textdegree}}1 {~}{{\textasciitilde}}1
}

\title{Rapport Exercice 4}
\author{Sanna Thomas, L3STI}
\date{\today}

\setlength{\parindent}{1cm}

\begin{document}

\maketitle

\section*{Introduction de l'exercice 4}

Les couleurs sont souvent décomposées en composantes rouge, verte et bleue en infor-
matique, et représentées par une série de réels compris entre 0 et 1 (inclus).

\section*{Question 1 --- Représentation des couleurs avec double}

En utilisant des doubles pour représenter les couleurs, chaque composante de couleur est représentée par un double de 64 bits (8 octets).

Donc, pour une couleur, nous avons 3 doubles $\times$ 8 octets = 24 octets.

\section*{Question 2 --- Représentation des couleurs avec des entiers}

En utilisant des entiers pour représenter les couleurs, chaque composante de couleur est représentée par un entier compris entre 0 et 255, soit 8 bits (1 octet).

Donc, pour une couleur, nous avons 3 entiers $\times$ 1 octet = 3 octets, soit 8 fois moins que la représentation avec des doubles. Cependant, la précision est moindre.

\section*{Question 3 --- Fonction de conversion et empaquetage dans un entier}

L'objectif est de convertir les composantes de couleur en entiers et de les empaqueter dans un seul entier.

\subsection*{Variables}

\texttt{r}, \texttt{v}, \texttt{b}: les composantes de couleur en entier, comprises entre 0 et 255.

\subsection*{Fonction de paquetage}

La fonction de paquetage prend les composantes de couleur en entier et les empaquette dans un seul entier. Elle vérifie tout d'abord que les composantes sont bien comprises entre 0 et 255, puis les empaquette dans un seul entier en utilisant des opérateurs bitwise tels que $\ll$ (Décalage à gauche), pour mettre côte-à-côte les différents octets, et $\vert$ (OU inclusif logique).

\begin{lstlisting}[caption=Fonction de Paquetage]

def empaquetage(r:int, v:int, b:int)->int:
"""
Empaquetage de couleurs dans un entier 

Args:
    r (int): la couleur rouge, comprise entre 0 et 255.
    v (int): la couleur verte, comprise entre 0 et 255.
    b (int): la couleur bleue, comprise entre 0 et 255.

Returns:
    int: Empaquetage sous forme d'entier
"""
assert 0 <= r <= 255 and 0 <= v <= 255 and 0 <= b <= 255, "Les couleurs doivent être comprises entre 0 et 255"

return b | (v << 8) | (r << 16)

\end{lstlisting}

\break\subsection*{Fonction de dépaquetage}

La fonction de dépaquetage prend un entier empaqueté et le décompose en ses composantes de couleur en entier. Elle utilise des opérateurs bitwise tels que $\gg$ (Décalage à droite) et $\&$ (ET logique) pour extraire les différents octets de l'entier empaqueté.

\begin{lstlisting}[caption=Fonction de Paquetage]

def depaquetage(paquet:int)->tuple[int, int, int]:
"""
Dépaquetage 

Args:
    paquet (int): La couleur sous forme de paquet à convertir

Returns:
    tuple[int, int, int]: Retourne les entiers des couleurs rouge, bleue et verte
"""
r = (paquet >> 16) & 0b11111111 # On réduit le nombre à 8 bits pour obtenir la couleur rouge
v = (paquet >> 8) & 0b11111111
b = (paquet) & 0b11111111

return r, v, b

\end{lstlisting}

\subsection*{Exemple et Output}
\subsubsection*{Explication étape par étape}
Supposons que nous avons les composantes de couleur suivantes:
\begin{itemize}
  \item Rouge: 122
  \item Vert: 22
  \item Bleu: 244
\end{itemize}

\textbf{Étape 1: définition des couleurs}
\begin{itemize}
  \item Rouge: 122 en binaire (1 octet): '01111010'
  \item Vert: 22 en binaire (1 octet): '00010110'
  \item Bleu: 244 en binaire (1 octet): '11110100'
\end{itemize}

\textbf{Étape 2: Empaquetage des couleurs dans un entier}
\begin{itemize}
  \item Empaquetage: '01111010(r) 00010110(v) 11110100(b)' en binaire (24 bits) = 8001268 en décimal.
\end{itemize}

\textbf{Étape 3: Dépaquetage de l'entier}
\begin{itemize}
  \item Dépaquetage de 8001268 en entiers: Rouge: 122, Vert: 22, Bleu: 244.
\end{itemize}

\subsubsection*{Exemple d'éxécution}
\begin{lstlisting}[caption=Exemple d'éxécution]
r, v, b = (122, 22, 244)
print(empaquetage(r, v, b))
print(depaquetage(empaquetage(r, v, b)))
\end{lstlisting}

\begin{lstlisting}[caption=Output]
>>> 8001268
>>> (122, 22, 244)
\end{lstlisting}


\end{document}