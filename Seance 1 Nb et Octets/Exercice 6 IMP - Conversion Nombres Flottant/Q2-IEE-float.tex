\documentclass{article}
\usepackage[utf8]{inputenc}
\usepackage{amsmath}
\usepackage{amsfonts}
\usepackage{amssymb}
\usepackage{graphicx}
\usepackage{indentfirst}
\usepackage{listings}
\usepackage{xcolor}
\usepackage{fancyhdr} % permet de creer des entetes et pieds de page


% Configuration de l'en-tête et du pied de page
\pagestyle{fancy}
\fancyhf{}
\lhead{Sanna Thomas}
\rhead{L3 STI}
\chead{Exercice 6 Q.2: Conversion IEE vers flottant}
\rfoot{Page n°\thepage}

\title{Rapport Exercice 4}
\author{Sanna Thomas, L3STI}
\date{\today}

\setlength{\parindent}{1cm}

\begin{document}

\maketitle

\section*{1. 0 10001000 00100101001000010000000}
\subsection*{Signe}
Le bit de signe est 0, donc le nombre est positif.
s = 1

\subsection*{Exposant}
L'exposant est 10010000 en binaire, soit 16 + 128 = 144 en décimal. 

e = 136

\subsection*{Mantisse}
La mantisse est 0.0100101001000010000000 en binaire

La mantisse est donc $2^{-3} + 2^{-6} + 2^{-8} + 2^{-11} + 2^{-16}$ = 0.14503479003 en décimal.

\subsection*{Nombre flottant}

\[x = s \times 2^{e-127} \times (1+m) \]
\[ x = 1 \times 2^{9} \times (1+2^{-3} + 2^{-6} + 2^{-8} + 2^{-11} + 2^{-16})\]
\[ x = 586.2578125 \]


\break\section*{2. 1 01111110 01110000000000000000000}
\subsection*{Signe}
Le bit de signe est 1, donc le nombre est négatif.
s = -1

\subsection*{Exposant}
L'exposant est 01111110 en binaire, soit 2 + 4 + 8 + 16 + 32 + 64 = 126 en décimal. 

\[e = 126\]

\subsection*{Mantisse}
La mantisse est 0.01110000000000000000000 en binaire

\[ m = 2^{-2} + 2^{-3} + 2^{-4} \]

\subsection*{Nombre flottant}

\[x = s \times 2^{e-127} \times (1+m) \]
\[ x = -1 \times 2^{-1} \times (1+2^{-2} + 2^{-3} + 2^{-4})\]
\[ x = -0.71875 \]


\end{document}