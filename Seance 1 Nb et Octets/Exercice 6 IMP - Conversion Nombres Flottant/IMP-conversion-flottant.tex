\documentclass[titlepage]{article}
\usepackage{listings}
\usepackage{amsmath} % Pour les environnements mathématiques avancés

\begin{document}

\begin{titlepage}
    \title{Réseaux et Sécurité: Conversion de nombres en flottant}
    \author{SANNA Thomas}
    \date{\today}
    \maketitle
\end{titlepage}

\section{Conversion de 0.625 en flottant}

\subsection{Formule}

La formule générale pour convertir un nombre en flottant est:
\[ 
x = s \times 2^{(e-127)} \times (1+m) 
\]
où:
- \( s \) est le signe (0 pour positif, 1 pour négatif)
- \( e \) est l'exposant
- \( m \) est la mantisse

\subsection{Décomposition}

Décomposons le nombre 0.625:
\[ 
0.625
\]
\[ 
= 0.5 + 0.125 
\]
\[ 
= \frac{1}{2} + \frac{1}{8} 
\]
\[ 
= \frac{1}{2} + \frac{1}{2^3} 
\]

\subsection{Factoriser par la plus grande puissance de 2}

Nous factorisons par la plus grande puissance de 2:
\[ 
2^{-1} + 2^{-3} 
\]
\[ 
= 1 \times 2^{-1} + 2^{-2} \times 2^{-1} 
\]
\[ 
= 2^{-1} \times (1 + 2^{-2}) 
\]
\[ 
= 2^{126-127} \times (1 + 2^{-2}) 
\]
\[ 
= 1(s) \times 2^{126-127} \times (1 + 2^{-2}) 
\]

\subsection{Résultat}

Le résultat final est:
\[ 
0.625 =  0(s) \ 01111110(e) \ 01000000000000000000000(m) 
\]

% Ajouter un espace vertical ici
\vspace{2cm}

\section{Conversion de -34,06125 en flottant}

\subsection{Formule}

La formule générale pour convertir un nombre en flottant est:
\[ 
x = s \times 2^{(e-127)} \times (1+m) 
\]
où:
- \( s \) est le signe (0 pour positif, 1 pour négatif)
- \( e \) est l'exposant
- \( m \) est la mantisse

\subsection{Décomposition avec valeur absolue du nombre}

Nous commençons par la décomposition de la valeur absolue du nombre. Nous mettrons \( s = -1 \) plus tard:
\[ 
-34,06125 
\]
\[ 
= 100010,0000 \ 1111 \ 1010 \ 1110 \ 0001 \ 01 
\]
\[ 
= 2^5 + 2^1 + 2^{-5} + 2^{-6} + \ldots 
\]

\subsection{Factoriser par la plus grande puissance de 2}

Nous factorisons par la plus grande puissance de 2:
\[ 
= 2^5 \times (1 + 2^{-4} + 2^{-10} + \ldots) 
\]
\[ 
(5 = e - 127 
\]
\[ 
e = 132) 
\]
\[ 
= (-1) \times 2^{132-127} \times (1 + 2^{-4} + 2^{-10} + \ldots) 
\]
\[ 
m = (2^{-4} + 2^{-10} + \ldots) 
\]

Ainsi, la représentation en flottant est:
\[ 
\rightarrow 1 \ 10000100 \ 00010000111110101110000 
\]

\subsection{Résultat}

Le résultat final est:
\[ 
-34,06125 = 1 \ 10000100 \ 00010000111110101110000 
\]

\end{document}